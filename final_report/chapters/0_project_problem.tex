\chapter{Project problem}\label{ch:project-problem}
Cloud computing is a rapidly growing service with competition from Google, Amazon, Microsoft and more that aims to
allow users to run computer programs that are too large, difficult or time consuming for users to run locally.
These services provide the computational resources, e.g.\ cpu cores, RAM, hard drive space, bandwidth, etc
to be able to run such programs. However, as these resources are limited, bottlenecks can occur when
numerous users all require large amounts of these resources, limiting the number of tasks
\footnote{Tasks, Programs and Jobs will be used interchangeable to refer to the same idea of a computer programs that
has a fixed amount of resources required to compute.} that can be run on the cloud servers simultaneously.

For Google Cloud Services (GCP), Microsoft Azure or Amazon Web Services, their cloud computing facilities contain huge
server nodes limiting the probability of such a bottleneck from occurring. And if such an event does occur users also
have a range of data centres across the global to use instead if a single data centre does becomes overloaded.
Therefore this work considers still a developing paradigm~\citep{mobile_edge_survey} called Edge/Mobile cloud computing.
Edge cloud computing is believed to have a wide range of application where traditional cloud computing would be
impractical. This could be due to tasks being highly delay-sensitive, intermittent internet connectivity
or high operational security that prevent or limit the effectiveness of traditional cloud computing.

Currently disaster response~\citep{mobile_edge_disaster}, smart cities~\citep{smart_disaster_management} and
internet-of-things~\citep{mobile_edge_IoT} are all area that utilise edge cloud computing due to its ability
to process computationally small tasks locally with low latency. For example, in smart cities, this
allows for smart intersection systems using of road-side sensors or smart traffic lights based
on cameras to minimise the waiting times~\citep{smart_cities_traffic_lights}. Or for the police to analysis
CCTV footage to spot suspicious behaviour or to track people between cameras~\citep{Sreenu2019}. In the case
of disaster response, maps can be produced using data from autonomous vehicles sensors that can then be used in the
search for potential victims and support responders~\citep{smart_disaster_management}.

However the problem of bottlenecking is of particular relevant in edge cloud computing, as instead of large server farms
that can be geographically distant from the users. Edge cloud computing server are significantly smaller, possibly
just high powered desktop computers and single server nodes. This results in greater demand on server resources,
meaning that efficient allocation of resources is extremely important. Because of this, resource allocation in edge
cloud computing is an important and interesting research area within edge cloud computing.

However it is believed that there are shortcoming in existing work about resource allocation within edge cloud
computing~\citep{vaji_infocom, Bi2019} due to the nature of how task resource usage is determined. Traditionally,
a user would submit a request for a fixed amount of resources, i.e.\ 2 cpu cores, 8Gb of ram, 20Gb of storage, that
would be allocated for the user. As a result, these resources can be redistributed till the user finishes with them.
This resource allocation mechanism is effective for cloud computing due to its simplicity for the user for  deciding
resource requirements, can utilise a linear pricing mechanisms and services having large resource pool making
bottlenecking rare.

Therefore in previous work by this author~\citep{FlexibleResourceAllocation} a novel resource allocation mechanism was
proposed to allow for significantly more flexibility in determining resource usage with the aims of reducing possible
bottlenecking. Mechanism is based on the principle that the time taken for an operation to complete is generally
proportional to the resources provided for the operation. An example for this is downloading an image, the time taken
is proportional to the bandwidth allocated. This sort of flexibility is similarly true for computing a
task~\footnote{This is not always true for all computational tasks, however for this work we presume this and leave
this case to future work} or sending results back to the user as well. Then by requesting that users provide the task's
total resource usage over its lifetime instead of the requested resource usage. Using this information task resource
usage is determined by the server rather than the user. Therefore using a deadline, provided by the user, it is possible
to reallocate resources around tasks to reduce the overall strain on certain resources while still finishing the task
with its deadline. Using this alternative resource allocation mechanism, it was found that results could achieve
20\% better than traditional resource allocation mechanisms in static cases investigated
in~\cite{FlexibleResourceAllocation}. This is due to the ability to proper balancing of resources, preventing
bottlenecks occurring as often, that in turn allowed more tasks to run simultaneously and to reduce the price for user
to run a task.

But in this previous work~\citep{FlexibleResourceAllocation}, the flexible resource allocation mechanism was only
considered in a static or one-shot approach where all tasks were presented at the first time step. At which point tasks
would be auctioned and resource allocated. As a result, practically the proposed algorithms would require tasks to
be processed in batches such that servers would bid on all tasks submitted every 5 minutes for example. Therefore
previous work could also not dynamically change the resources allocated between batches making it impractical to be
used commercially, this work aims to address these problems in previous work.

These problems are addressed by introducing time into the optimisation problem (outlined in
section~\ref{sec:optimisation-problem}) but due to this addition, all previous mechanism proposed
in~\cite{FlexibleResourceAllocation} are incompatible with the new optimisation problem. Therefore using a standard
auction mechanism, this project investigates different methods of learning how to bid on tasks based on their resource
requirements and to efficiently allocate resources to tasks by a server.

This report is set out is following chapters. Chapter~\ref{ch:proposed-solution-to-problem} proposed a solution
to the project outline in this chapter with chapter~\ref{ch:justification-of-the-solution} justifying why this
approach as taken over alternative. Chapter~\ref{ch:background-literature} investigates the previous research
that this project builds upon within both resource allocation in cloud computing and reinforcement learning methods.
The proposed solution is then implemented in chapter~\ref{ch:implementation-of-the-solution} with testing and
evaluation in chapters~\ref{ch:testing-of-the-implementation} and~\ref{ch:evaluation-of-the-implementation} respectively.

In addition to this report, the paper referred to as~\cite{FlexibleResourceAllocation}, with a copy in
appendix~\ref{app:aamas_paper}, was completed within this academic year and thus consider part of this project's work.
In addition to this paper, the work was presented at SPIE Defense + Commercial Sensing 2020 as a recorded digital
presentation with a copy of the slides in appendix~\ref{app:spie_presentation} and a link provided for the recording.

