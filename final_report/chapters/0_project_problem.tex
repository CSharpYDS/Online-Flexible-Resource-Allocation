\chapter{Project problem}\label{ch:project-problem}
Cloud computing is a rapidly growing service with competition from Google, Amazon, Microsoft and more that aim to
allow users to run computer programs that are too large, difficult or time consuming for users to run locally.
These services provide the computational resources, e.g.\ cpu cores, RAM, hard drive space, bandwidth, etc
to be able to run such programs. However, as these resources are limited, bottlenecks can occur when
numerous users all require large amounts of these resources, which limits the number of tasks
\footnote{Tasks, Programs and Jobs will be used interchangeable to refer to the same idea of a computer programs that
has a fixed amount of resources required to compute the program.} that can be run on the cloud services simultaneously.

For Google and the like, their cloud computing facilities contain vast server nodes limiting the probability of such
bottlenecking occurring and if such an event does occur, users have a range of data centres across the global to use instead.
Therefore this work considers still a developing paradigm~\citep{mobile_edge_survey} called Edge/Mobile cloud computing.
Edge cloud computing is believed to have a wide range of application where using cloud computing would traditionally
be impractical. This could be due to tasks being highly delay-sensitive, intermittent internet connectivity
or high operational security that prevent or limit the effectiveness of traditional cloud computing.

Currently: disaster response, smart cities and internet-of-things (IoT) are all popular technologies that utilise
edge cloud computing due to its ability to process programs locally with low latency. For smart cities, this
allows for the possibly of smart intersections with the use of road-side sensors or smart traffic lights based
on cameras to minimise the waiting times~\citep{smart_cities_traffic_lights}. Or for the police to analysis
CCTV footage to spot suspicious behaviour or to track people between cameras~\citep{Sreenu2019}. In the case
of disaster response, maps can be produced using autonomous vehicles sensors to be used in the search for potential
victims and support responders~\citep{smart_disaster_management}.

However the problem of bottlenecking is of particular relevant in edge cloud computing, as instead of large server farms
that can be geographically distant from the users. Edge cloud computing server are significantly smaller than Google
style data centers and can be just high powered desktop computers. This results in the demand on resources to be much
greater, requiring that servers try to allocate their resources was efficiently as possible.
Because of this, resource allocation in edge cloud computing is an important and interesting research area.

However it is believed that there is a shortcoming of existing work in resource allocation within
edge cloud computing~\citep{vaji_infocom, Bi2019} due to the nature of task resource requirements. Traditional, to
compute these task, several types of resource are required including communication bandwidth, computational power and
data storage resources~\cite{vaji_infocom} that a user would request a fixed amount of each of these resources.
This thus requires the server to allocate such resources to the user however anyone user from accessing the resources
till the user is finished with them. However this has the disadvantage bottlenecking occurring due to the inflexible
nature of resources that cant allow balancing of resources between users. The reason for the continued use of such
methods is that in traditional cloud computing such bottlenecking is rare and it allows for simpler pricing mechanism
as the price can be proportional to the quantity of resources required.

Previous work by this author~\citep{FlexibleResourceAllocation} proposed a novel resource allocation method to allow
for significantly more flexibility aims that this would reduce bottlenecking of resources. This was done by requesting
that users provide a total resource usage over a tasks lifetime instead of the required resource usage. With this
change how much of each resource to be allocated could be decided by the server instead of the user. This is possible
as the time taken for data to be send and received is proportional to the allocated bandwidth. This sort of flexibility
is similarly true for computing a task as well. Therefore using a deadline, provided by the user, it is possible
to reallocate resources around to reduce the overall strain on certain resources. Therefore Using this alternative
resource allocation procedure, bottlenecks can be prevent through proper balancing of resources that in turn will allow
more tasks to run simultaneously and to reduce the price for user to run a task.

However this work only considered a static or one-shot approach where all tasks were presented at the first time step.
At which point tasks would be auctioned and resource allocated. As a result while tasks would be processed in batches
such that tasks would bid on all tasks submitted every 5 minutes or so. Therefore previous work could also not
dynamically change the resources allocated between batches, this work aims to address these problems in previous work.

These problems are addressed by introducing time into the optimisation problem (outlined in section~\ref{}) but due
to this additional all previous mechanism used are incompatible with the new optimisation problem. Therefore using
a standard auction mechanism, this project investigates different methods of learning how to bid on tasks
based on their resource requirements and to efficiently allocate resources to tasks by a server.

This report is set out is following chapters. Chapter~\ref{ch:proposed-solution-to-problem} proposed a solution
to the project outline in this chapter with chapter~\ref{ch:justification-of-the-solution} justifying why this
approach as taken over alternative. Chapter~\ref{ch:background_literature} investigates the previous research
that this project builds upon within both resource allocation in cloud computing and reinforcement learning methods.
The proposed solution is then implemented in chapter~\ref{ch:implementation-of-the-solution} with testing and
evaluation in chapters~\ref{ch:testing-of-the-implementation} and~\ref{ch:evaluation-of-the-implementation} respectively.

%% Todo explain what the project does

%% Unneeded text
%has the assumption that tasks have fixed resource requirements. However, flexibility is possible in practise
%with how resources are allocated to each task. For example, the allocated bandwidth for loading the program is
%proportional to the time taken to load the program. This is true of also the computational requirements and
%for sending results back to the user. This project investigates flexible allocation of resource and pricing
%mechanisms when task arrive over time and have private values.
%
%To compute these task, several types of resources are required included communication bandwidth, computational power
%and data storage resources~\citep{vaji_infocom}. Tasks also contain a deadline such that the program must be completed
%before this point. Task will have a private monetary value that depends on the program itself and its value to the owner,
%.e.g analysis air pollution is less important than preventing traffic jams at rush hour or tracking a criminal on the
%run. This project is interested in allocated task to servers to maximise the social welfare (sum of all allocated
%task values) over time. But due to users being self-interested, they may behave strategically~\citep{Bi2019} or prefer
%to not reveal their value publicity~\citep{Pai2013}.
%
%By building upon the flexible resource allocation method \footnote{This method is explain in greater detail in
%section~\ref{ch:proposed-solution-to-problem}.} and with the use of deep reinforcement learning agents, this project
%explores the effectiveness of different reinforcement learning methods and compares to non-learning agents.
%
%This project considers the case where users state their total resource requirements for a program, instead of the
%standard procedure that users request a fixed amount of resources for their program. This change now allows the cloud
%provider the ability to balance resource demand as it had completed knowledge of user's requirements. With this
%knowledge, a task's allocated resources can be dynamically changed as new tasks arrives or finish. Using this
%alternative resource allocation procedure, bottlenecks can be prevent through proper balancing of resources that in
%turn will allow more tasks to run simultaneously and can also lower the price to run a task.
