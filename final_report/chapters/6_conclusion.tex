\chapter{Conclusion and future work}\label{ch:conclusion-and-future-work}
The aim of this project was to expand previous research to fix perceived flaws in the formulation by introducing the
notion of time into the resource allocation optimisation model. As a result, a new optimisation problem was presented
in Section~\ref{sec:optimisation-problem} with an auction mechanism proposed as well to deal with self-interested
users and to distribute tasks to self-interested servers. To know how to efficiently bid and allocation resources to
tasks, reinforcement learning agents were proposed that aimed to learn these policies. An implementation of an MEC
environment was developed and numerous reinforcement learning algorithms were used to train both auction and resource
weighting agent. These agents were found to efficiently learn an optimal policy however were found to not produce
optimal policies such that \~5\% of all tasks were no completed within there time frame. A range of reasons why this
may have occurred in Chapter~\ref{ch:testing-and-evaluation} as a well as policy gradient agents being unable to
escape local maxima prevent them from achieving results close to that of the deep Q learning agents. Therefore this
project has been viewed as a success however this author believes have more research and analysis of agents is required
before such agents can be implemented into real-life systems. \\
For future work into this project, this author believes that several additions to the agents proposed could greatly
improve their performance like n-step rewards~\citep{multi-step-dqn} and distributional agents~\citep{distributional_dqn}
that would improve Q value estimation within stochastic environment. An additional heuristic for the policy gradient,
would be use a centralised critic~\citep{maddpg} that has been proposed in mix competitive-cooperative environment to
help agents work together.

The word count of the project can be found in Appendix