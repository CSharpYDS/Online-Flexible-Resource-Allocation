\chapter{Justification of the solution}\label{ch:justification-of-the-solution}
The proposed solutions to the problem were outlined in chapter~\ref{ch:proposed-solution-to-problem}, this chapter
explains the reasoning for the chooses made for the auction mechanism in
section~\ref{sec:justification-of-vickrey-auction-mechanisms} and the network architecture in
section~\ref{sec:justification-of-auctioning-and-resource-allocation-network-architectures}.

\section{Justification of Vickrey auction mechanisms}\label{sec:justification-of-vickrey-auction-mechanisms}
In auction theory, there are numerous types of auctions that have different properties and uses in different areas.
The area in which this project is interested in is single indivisible items as while the item has multiple resource
requires, a server is required to buy the task as a single unit. Therefore combinatorial auctions are not considered
here. Table~\ref{tab:auctions_descriptions} outlines a description of possible applicable auction mechanisms while
table~\ref{tab:auction_properties} outline the properties of each auction mechanisms in the previous table.

\begin{longtable}{|p{3cm}|p{10cm}|} \hline
    \textbf{Auction type} & \textbf{Description} \\ \hline
    English auction & A traditional auction where all participant can bid on a single item with the price slowing ascending till
    only a single participant is left who pays the final bid price. Due to the number of rounds, this requires a large amount of
    communication and requires tasks to be auctioned in series. \\ \hline

    Dutch auction & The reverse of the English auction where the starting price is higher than anyone is willing to pay with the price
    slowly dropping till the first participant "jumps in". This can result in sub-optimal pricing if the starting price is not highest enough
    and the latency can have a large effect on the winner. \\ \hline

    Japanese auction & Similar to the English auction except that the auction occurs over a set period of time with the last highest
    bid being the winner. This means that it has the same disadvantages as the English auction except that there is no guarantee
    that the price will converge to the maximum. Plus additional factors like latency can have a large effect on the winner that
    will have a larger affect in the application of this project, edge cloud computing.
    But this time limit results in the auction taking a fixed amount of time unlike the English or Dutch auctions. \\ \hline

    Blind auction & Also known as a First-price sealed-bid auction, all participants submit a single secret bid for an item with the highest bid winning and
    pays their bid value. As a result there is no dominant strategy (not incentive compatible) as an agent would not wish to bid higher than their task evaluation
    but if all other agents bid significantly lower then it would have been beneficial for the agent to bid much lower than their true evaluation.
    Due there being a single round of biding, latency doesn't affect an agent and many more auctions could occur within the same time a English,
    Dutch or Japanese auction would take to run. \\ \hline

    Vickrey auction~\citep{vickrey} & Also known as a second-price sealed bid auction, all participants submit a single secret bid
    for an item with the highest bid winning but it only pays the price of the second highest bid. Because of this, it is a dominant
    strategy for an agent to bid its true value as even if the bid is much higher than all other participants its doesn't matter. \\ \hline

    \caption{Descriptions of auctions}
    \label{tab:auctions_descriptions}
\end{longtable}

\begin{table}[h]
    \centering
    \begin{tabular}{|l|c|c|c|} \hline
        Auction & Incentive compatible & Iterative & Fixed time length\\ \hline
        English & False & True & False \\ \hline
        Japanese & False & True & True \\ \hline
        Dutch & False & True & False \\ \hline
        Blind & False & False & True \\ \hline
        Vickrey & True & False & True \\ \hline
    \end{tabular}
    \caption{Properties of the auctions described in Table~\ref{tab:auctions_descriptions}}
    \label{tab:auction_properties}
\end{table}

Due to the properties that the Vickrey auction has in compared to the other auctions, of incentive compatibility and
single round being of most importance means that this auction is believe the most appropriate for this project. The
auction has additional properties for learning with agents due to the auction being strategyproof meaning that hte
dominant strategy is to truthfully bid the agent's value of the task. This has the advantage of allowing agent to
self-train as the agents don't need to learn how to out price other agents. Another advantage of the auction is that
it is not iterative, making the auction fast with only a single round of bidding required. This means that the auction
can be certain to complete within a fixed amount of time as server must submit a bid within the time frame or loss out
on the task.

%% Todo justification that reserve vickrey auction is still incentive compatible

%This is as the modification required for an auction mechanism to be used is not to the auction itself but to the way
%that server price tasks. In traditional cloud computing, it was possible to create combinations of resources that means
%that users just buy the minimum amount of resources for their task with the price determined by the amount of
%resources requested. But due to the flexible resource allocation mechanism this style of linear pricing strategy is not
%effective as the demand on resources is not easy to calculate and the resource usage by other tasks is difficult to
%predict in the future. If an user wishes to run on task on the cloud, the task can be put forward with its
%requirements of required storage, computation, results data and deadline.

\section{Justification of Auctioning and Resource allocation network architectures}\label{sec:justification-of-auctioning-and-resource-allocation-network-architectures}
Neural network can be made up an unlimited number of layers therefore the choice of layer to allow for efficient and
fast network training. In table~\ref{tab:neural_network_layers}, a range of primary layer of agent neural networks
are explored and explained with there difference.

\begin{longtable}{|p{3.5cm}|p{11cm}|} \hline
    \textbf{Neural Network} & \textbf{Description} \\ \hline
    Artificial neural networks~\citep{ANN} & Originally developed as a theoretically approximation for the brain, it
        was found that for networks with at least one hidden layer that a neural network could approximate any
        function~\citep{csaji2001approximation}. This made neural networks extremely helpful for cases where it would
        normally be difficult for a human to specify the exact function, artificial neural network can be trained
        through gradient descent to find a close approximation to the true function. \\ \hline

    Recurrent neural network~\citep{RNN} & A major weakness of artificial neural networks is that it must use a fixed
        number of inputs and outputs making it unusable with text, sound or video where previous data is important
        for understanding the inputs. Recurrent neural network's extend neural networks to allow for connections to
        neurons again so that the network is not stateless. This means that individual letters of a words can be
        passed in with the network "remembering" the previous letter. \\ \hline

    Long/Short Term Memory~\cite{LSTM} & While recurrent neural network's can "remember" previous inputs to the
        network, it also struggles from the vanishing or exploding gradient problem where gradient tends to zero or
        infinity making it unusable. This network aims to prevent this by using forget gates that determines how much
        information the next state will get, allowing for more complexity information to be learnt compared to
        recurrent neural networks. \\ \hline

    Gated Recurrent unit~\cite{GRU} & Gated recurrent unit are very similar to long/short term memory, except for the
        use of a different wiring mechanisms and use one less gate with an update date instead of forgot gates.
        These changes mean that gated recurrent units allow for them to run faster and are easier to code than
        long/short term memory however are not as expressive allowing for less complex functions to be encoded. \\ \hline

    Neural Turing Machine~\cite{NTM} & Inspired by computers, neural turing machines build on long/short term memory
        by using an external memory module that instead of memory being inbuild to the network. This allows for
        external observers to understand what is going on much better than other networks due to their
        black-box nature. \\ \hline

    Differentiable neural computer~\cite{DNC} & This is an expansion to the neural turing machine that allows the
        memory module to scalable in size allowing for additional memory to be added if needed. \\ \hline
    \caption{Neural network layer descriptions}
    \label{tab:neural_network_layers}
\end{longtable}

\subsection{Justification for Auctioning networks}\label{subsec:justification-for-auctioning-networks}
Outlined in Table~\ref{tab:neural_network_layers} is the properties of popular neural network layer architectures that
would allow for a variable amount of inputs (except for artificial neural networks). This is due to there being
multiple inputs with each of the current allocated tasks to a server that till compute time is of unknown length. Of
the available architecture, long/Short term memory model is the simplest model but still with the complexity to encode
the policy. With the neural turing machine and differentiable neural network, these networks are extremely complex and
require a large amount of data to train the networks. Also the ability of these networks to be able to store data in
external storage is not important as the data doesn't need to be store for future inputs. The opposite problem exists
for the recurrent neural network or the gated recurrent unit that they are possibly not complex enough to encode the
policy.

\subsection{Justification for Resource allocation networks}\label{subsec:justification-for-resource-allocation-networks}
The justification for the resource allocation agent neural network is very similar justification to the previous
subsections~\ref{subsec:justification-for-auctioning-networks}. The long/short term memory architecture should be
complex enough but it is possible that the ability to use external storage of neural turing machine and differentiable
neural network to store the allocation of resource to previous tasks. It is believed that this additional complexity
will not allow for the heuristic to do better but it could be investigated in future work.

The reason for the output to be the resource weighting rather than the actual resources is it would require the network
to learn a less complex function in comparison. This means that the network learns instead how important the allocation
of resources are for a task instead of the exact amount of resources allocated.