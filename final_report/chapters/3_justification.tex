\chapter{Justification of the solution}\label{ch:justification-of-the-solution}
The proposed solutions for the project is outlined in chapter~\ref{ch:proposed-solution}. This chapter
explains the reasoning for the chooses made for the solution with regards to the auction mechanism in
section~\ref{sec:justification-of-auction-mechanisms} and the proposed agents in
section~\ref{sec:justification-of-auctioning-and-resource-allocation-agent}.

\section{Justification of auction mechanisms}\label{sec:justification-of-auction-mechanisms}
In mechanism design, there are numerous types of auctions, each with different properties and applications.
The types of auction that this project is interested in is single indivisible items where an item is sold as a single
units. This is required as while the task has multiple required resources, a server is required to buy the task as a
whole with a single price and can't be split up as in combinations auctions~\citep{combinational_auctions}.
Table~\ref{tab:auctions_descriptions} outlines a description of possible applicable auction mechanisms while
table~\ref{tab:auction_properties} outline the properties of each auction mechanisms in the previous table.

\begin{longtable}{|p{3cm}|p{10cm}|} \hline
    \textbf{Auction type} & \textbf{Description} \\ \hline
    English auction & A traditional auction where all participant can bid on a single item with the price slowly
        ascending till only a single participant is left who pays the final bid price. Due to the number of rounds,
        this requires a large amount of communication. \\ \hline

    Dutch auction & The reverse of the English auction, where the starting price is higher than anyone is willing to
        pay with the price slowly dropping till the first participant "jumps in". This can result in sub-optimal pricing
        if the starting price is not highest enough or a large number of rounds is required till anyone bids. Plus due
        the auctions occurring over the internet, latency can have a large effect on the winner. \\ \hline

    Japanese auction & Similar to the English auction except that the auction occurs over a set period of time with the
        bid increasing over time and last highest bid being the winner. This means that it has the same disadvantages
        as the English auction except that there is no guarantee that the price will converge to the maximum. Plus
        additional factors like latency can have a large effect on the winner and resulting price. But due to the time
        limit, it has a known amount of time till it finishes unlike the English or Dutch auctions. \\ \hline

    Blind auction & Also known as a First-price sealed-bid auction, all participants submit a single secret bid for an
        item with the highest bid winning. As a result there is no dominant strategy (not incentive compatible) as an
        agent would wish to bid only a small amount more than the next highest price in order overpay for the item.
        But due to there being only a single round of biding, latency doesn't affect an agent and allows many
        more auctions could occur within the same time compared to the English, Dutch or Japanese auctions. \\ \hline

    Vickrey auction~\citep{vickrey} & Also known as a second-price sealed bid auction, participants each submit
        a single secret bid for an item with the highest bid winning like the blind auction. However the winner only
        pays the price of the second highest bid. Because of this, it is a dominant strategy (incentive compatible)
        for an agent to bid its true value as even if the bid is much higher than all other participants its doesn't
        matter as they pay the minimum required for them to win. \\ \hline
    \caption{Descriptions of feasible auctions for the project: English, Dutch, Japanese, Blind and Vickrey auction}
    \label{tab:auctions_descriptions}
\end{longtable}

\begin{table}[h]
    \centering
    \begin{tabular}{|l|c|c|c|} \hline
        Auction  & Incentive compatible & Iterative & Fixed time length\\ \hline
        English  & False                & True      & False            \\ \hline
        Japanese & False                & True      & True             \\ \hline
        Dutch    & False                & True      & False            \\ \hline
        Blind    & False                & False     & True             \\ \hline
        Vickrey  & True                 & False     & True             \\ \hline
    \end{tabular}
    \caption{Properties of the auctions described in Table~\ref{tab:auctions_descriptions}}
    \label{tab:auction_properties}
\end{table}

The advantage of using the Vickrey auction is that it is incentive compatible meaning that the dominant strategy for
bidding on a task is to bid your truthful value. This should help auction agents as they dont need to learn how to
outbid another agent, it only needs learn to efficiently evaluation each task. This also allows agents to learn through
self-play effectively as only agents are not required in order to "out play" other agents. The auction also has only a
single round of bidding compared to alternative auctions like English or Dutch auctions. This makes auctioning fast
no matter the number of servers. Due to the properties that the Vickrey auction has in comparison to the other auctions,
of incentive compatibility and only taking a single round to finish. The Vickrey auction is believe to be the most
appropriate for this project.

\section{Justification of Auctioning and Resource allocation agent}\label{sec:justification-of-auctioning-and-resource-allocation-agent}
There are a range of possibly neural network layers that allow for multiple inputs and a single output,
table~\ref{tab:neural_network_layers} outlines the most common of these layers. This information is used for both the
auction and resource allocation agents neural network architectures.

\begin{longtable}{|p{3.5cm}|p{10cm}|} \hline
    \textbf{Neural Network} & \textbf{Description} \\ \hline
    Artificial neural networks~\citep{ANN} & Originally developed as a theoretically approximation for the brain, it
        was found that for networks with at least one hidden layer, neural networks could approximate any
        function~\citep{csaji2001approximation}. This made neural networks extremely helpful for cases where it would
        normally be too difficult for a human to specify the exact function as they can be trained through gradient
        descent and supervised learning to find a close approximation to the true function. \\ \hline

    Recurrent neural network~\citep{RNN} & A major weakness of artificial neural networks is that it must use a fixed
        number of inputs and outputs making it unusable with text, sound or video where previous data is important
        for understanding the inputs. Recurrent neural network's extend neural networks to allow for connections to
        previous neurons to "pass on" information. However recurrent neural networks struggle from vanishing or
        exploding gradient during training. \\ \hline

    Long/Short Term Memory~\citep{LSTM} & While recurrent neural network's can "remember" previous inputs to the
        network, it also struggles from the vanishing or exploding gradient problem where gradient tends to zero or
        infinity making it unusable. LSTM aim to prevent this by using forget gates that determines how much
        information the next state will get, allowing for more complexity information to be learnt compared to
        recurrent neural networks. \\ \hline

    Gated Recurrent unit~\citep{GRU} & Gated recurrent unit are very similar to long/short term memory, except for the
        use of a different wiring mechanisms and the one less gate, an update date instead of two forgot gates.
        These changes mean that gated recurrent units run faster and are easier to code than long/short term memory,
        however are not as expressive meaning that less complex functions can be encoded. \\ \hline

    Neural Turing Machine~\citep{NTM} & Inspired by computers, neural turing machines build on long/short term memory
        by using an external memory module instead of memory being inbuilt to the network. This allows for external
        observers to understand what is going on much better than other networks due to their black-box nature. \\ \hline

    Differentiable neural computer~\citep{DNC} & An expansion to the neural turing machine that allows the memory
        module to scalable in size allowing for additional memory to be added if needed. \\ \hline
    \caption{Neural network layer descriptions}
    \label{tab:neural_network_layers}
\end{longtable}

\subsection{Justification for Auction agent network architectures}\label{subsec:justification-for-auction-agent-network-architectures}
Outlined in Table~\ref{tab:neural_network_layers} are the properties of popular neural network layer architectures that
would allow for a multiple inputs (except for artificial neural networks) with a single output. Of the available
architecture, long/Short term memory model is believed to have the simplest model, that still has the complexity to
encode the optimal policy. With the neural turing machine and differentiable neural network, these networks are
extremely complex and require a large amount of data to train the networks. Also the ability of these networks to be
able to store data in external storage is not important as the data doesn't need to be store for future inputs. The
opposite problem exists for the recurrent neural network or the gated recurrent unit that they are possibly not complex
enough to encode the policy. Because of this, the LSTM network is believed to be the most appropriate network for the
proposed agents.

\subsection{Justification for Resource allocation agent network architecture}\label{subsec:justification-for-resource-allocation-agent-network-architecture}
The justification for the resource allocation agent neural network is very similar justification to the auction agent
from subsection~\ref{subsec:justification-for-auction-agent-network-architectures}. The long/short term memory
architecture should be complex enough but it is possible that the ability to use external storage of neural turing
machine and differentiable neural network to store the allocation of resource to previous tasks would be helpful.
However due to its complexity, it was implemented.

% TODO ADD SEQ2SEQ