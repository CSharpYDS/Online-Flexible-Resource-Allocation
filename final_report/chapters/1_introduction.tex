\chapter{Introduction}\label{ch:introduction}
Cloud computing is a rapidly growing technology with competition from Google, Amazon, Microsoft and others that aims to
allow users to run computer programs that are too large, difficult or time consuming to run locally.
These services provide the computational resources, e.g.\ CPU cores, RAM, hard drive space, bandwidth, etc
to be able to run such programs. However, as these resources are limited, if users request an unbalance quantity of
resources, bottlenecks can occur limiting the number of tasks~\footnote{Tasks, Programs and Jobs will be used
interchangeable to refer to the same idea of a computer programs that has a fixed amount of resources required to
compute.} that can be run on servers simultaneously.

For Google Cloud Services (GCP), Microsoft Azure or Amazon Web Services, their cloud computing facilities contain huge
server nodes limiting the probability that such a bottleneck occurs. But if such an event does occur, users
have a range of data centres across the global to use if a single data centre does becomes overloaded with requests.
Therefore this work considers a developing paradigm~\citep{mobile_edge_survey} called Mobile Edge
Computing~\citep{hu2015mobile} referred to as MEC. MEC aim to provide users with the ability to run their
tasks closer to them in the network, reducing latency, network congestion and providing better application performance.

Currently Disaster Response~\citep{mobile_edge_disaster}, Smart Cities~\citep{smart_disaster_management} and the
Internet-of-Things~\citep{mobile_edge_IoT} are all areas that utilise MECs due to its ability
to process computationally small tasks locally with low latency. For example, in Smart Cities, this
allows for smart intersection systems using road-side sensors or smart traffic lights to minimise cars waiting times
at traffic lights and reduce traffic congestion~\citep{smart_cities_traffic_lights}. Or for the
police to analyse CCTV footage to spot suspicious behaviour and to track people between cameras~\citep{Sreenu2019}.
In the case of Disaster Response, maps can be produced using data from autonomous vehicles' sensors that can support in
the search for potential victims and support responders in planning rescues~\citep{smart_disaster_management}.

With MECs, the problem of bottlenecks is of particular relevant as instead of large server
farms that can be geographically distant from the users. Servers are significantly smaller, possibly
just high powered desktop computers or single server nodes. This results in greater demand on individual server
resources, meaning that efficient allocation of these resources is of growing importance as the technology continues to
grow and be utilised by new technologies.

However it is believed that there are shortcomings in existing research about resource allocation within
MEC~\citep{vaji_infocom, Bi2019} due to the nature of how task resource usage is determined. Traditionally,
a user would submit a request for a fixed amount of resources, i.e.\ 2 CPU cores, 8GB of RAM, 20GB of storage, that
would be allocated for the user. As a result, these resources can't be redistributed until the user finishes with them.
The reason that this form of resource allocation is used and effective within cloud computing is due to its simplicity
for the user to decide resource requirements. The utilisation of simple linear pricing mechanisms and it being rare for
servers with large resource capacity to have bottlenecks. However it is believed that the problem of bottlenecks within
MEC systems, warrant the investigation of an alternative resource allocation mechanisms.

In previous work a novel resource allocation mechanism was proposed~\citep{FlexibleResourceAllocation} to allow for
significantly more flexibility in determining resource usage with the aims of reducing possible
bottlenecks. The mechanism is based on the principle that the time taken for an operation to complete is generally
proportional to the resources provided for the operation. An example for this is downloading an image, the time taken
is proportional to the bandwidth allocated. This sort of flexibility is similarly true for computing of most
tasks~\footnote{It is well known that some algorithm are not linearly scalable making this principle incompatible with
those tasks. Therefore in this work consider the case for algorithms that can be scalable linearly and leaves case of
non-linearly scalable tasks to future research.} or sending back results to the user. \\
Based on this principle, a modified resource allocation mechanism can be
reconstructed such that the users provide the task's total resource usage over its lifetime instead of the task's
requested resource usage. This allows for each task's resource usage to be determined by the server rather than the user
increasing a server's flexibility and control. Using this flexible resource allocation mechanism, algorithms proposed achieved 20\%
better social welfare than a fixed inflexible resource allocation mechanisms in one-shot cases investigated by~\cite{FlexibleResourceAllocation}.
This is due to the ability of the algorithms to properly balance task resources, preventing bottlenecks occurring as often, which in turn allowed
for more tasks to run simultaneously and to reduce the price.

However that work only considered the proposed mechanism within a one-shot case where all tasks were presented
at the first time step, where in all tasks would be auctioned and resource allocated. As a result, practically the proposed
algorithms would require tasks to be processed in batches, such that servers would bid on all tasks submitted every 5
minutes for example. This also meant that while resources could be dynamically allocated at the first time step, they
would not change during the next batches until the task was completed. This work aims to address these problems.

This was achieved by introducing time into the optimisation problem (outlined in section~\ref{sec:optimisation-problem}).
As a result, task now arrive over time and servers can redistribute resources at each time step. However, all
previous mechanisms proposed in~\cite{FlexibleResourceAllocation} are incompatible with this modified online flexible
optimisation problem. Therefore this work investigates Reinforcement Learning methods that train agents to optimally
bid on tasks based on their resource requirements and efficiently allocate resources to tasks running on a server.

This report is set out in the following chapters. Chapter~\ref{ch:literature-review} investigates previous research
that this project builds upon within both resource allocation in Cloud Computing and Reinforcement Learning.
Chapter~\ref{ch:optimising-resource-allocation-in-mec} proposes a solution to the problem outline in
Chapter~\ref{ch:introduction}.
The solution is implemented in chapter~\ref{ch:implementing-flexible-resource-allocation-environment-and-server-agents} with testing and
evaluation in Chapter~\ref{ch:testing-and-evaluation}.
Chapter~\ref{ch:conclusion-and-future-work} presents the conclusion along with future work for the project.

In addition to this report, the paper referred to as~\cite{FlexibleResourceAllocation} was written within this
academic year and thus considered part of this project's work. A copy of the paper can be found in
Appendix A. The paper was also presented at SPIE Defense and Commercial Sensing 2020 as a recorded digital presentation.
A copy of the slides can be found in Appendix B with a link to the recording.
