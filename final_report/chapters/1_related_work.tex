\chapter{Related Work}\label{ch:background-literature}
Due to the novel approach for resource allocation in cloud computing, there are few papers that consider flexible
resource allocation. However there is a considerable amount of research in the area of resource allocation and
pricing in cloud computing, some of which use auction mechanisms to deal with
competition~\citep{KUMAR2017234,Zhang2017,Du2019,Bi2019}.

Section~\ref{sec:related-work-in-cloud-computing} considers the related work in pricing and resource allocation in
edge cloud computing and how this project builds upon it providing novel advancements.
Section~\ref{sec:related-work-in-machine-learning} considers the recent work in the field of machine learning,
din particular reinforcement learning. This is important as it relates to the proposed solution and how this
project aims to solve the optimisation problem in section~\ref{sec:optimisation-problem}.

\section{Related Work in Cloud Computing}\label{sec:related-work-in-cloud-computing} %% Todo rewrite
A majority of the approaches for pricing and resource allocation in cloud computing require users to request a
fixed amount of certain resource with the cloud provider having no control over the resources only the servers that the
task was allocated to~\citep{KUMAR2017234,Zhang2017,Du2019,Bi2019}. The flexible approach that this project
assumed has only been considered in~\cite{FlexibleResourceAllocation} that allows the server to distribute its resources
more efficiently based on each task's requirements. The primary difference between this project and that paper is that this
project considers the addition of time allowing for resource speed to change over time and that there are task stages.

Previous work by~\cite{FlexibleResourceAllocation} considers three solutions to a single-shot problem case,
a greedy algorithm to quickly approximate a solution to maximise the social welfare and two auction mechanisms as server
are normally paid for usage of their resources. The greedy algorithm is a polynomial time algorithm that will find solution
within $\frac{1}{n}$ of the optimal social welfare.
This is done through the use of modular heuristics for ordering the task by density then for each task, select a server
based on available resource on each servers then to allocate resources that minimises a resource heuristics.
Using certain heuristics, the greedy algorithm achieves at least 90\% of the optimal solution and 20\% more than optimal
solution for fixed resource equivalent problems. A new distributed iterative auction was developed that use a reverse vcg
principle to calculate a task price that meant that a task didnt need to reveal its private value also that the
auction could be run in a decentralised way. This means that the auction is budget balanced however it is not
economically efficient or incentive compatible. The third algorithm is an implementation of a single parameter
auctions~\citep{nisan2007algorithmic_critical_value} using the greedy algorithm to find the critical value of a task.
Using this mechanism with a monotonic value density heuristic means that the auction is incentive compatible.

Other closely related work on resource allocation in edge clouds~\cite{vaji_infocom} considers both the placement of
code/data needed to run a specific task, as well as the scheduling of tasks to different edge clouds. The goal there
is to maximize the expected rate of successfully accomplished tasks over time. Our work is different both in the setup
and the objective function. Our objective is to maximize the value over all tasks. In terms of the setup, they assume
that data/code can be shared and they do not consider the elasticity of resources.

\section{Related Work in Machine Learning}\label{sec:related-work-in-machine-learning}  %% Todo write