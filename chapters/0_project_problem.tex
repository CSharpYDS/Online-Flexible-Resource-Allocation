\chapter{Project problem}\label{ch:project-problem}
Google Cloud Platform, Amazon Web Service and Microsoft Azure provide a service to users with computer programs that
are too large, difficult or time consuming to be run on standard computer. User can request a fixed amount of resources
to run the program, e.g.\ cpu cores, RAM, hard drive space, bandwidth, etc. However,
this can create bottlenecks on certain resources due to large numbers of resource requests preventing other jobs from
running. This problem is particularly relevant in edge cloud computing as servers
are small thus making the demand on resources much greater. This project considers the case where the user states
the total resource requirements for the program instead of the standard procedure that user request a fixed
amount of resources. This allows the cloud provider the ability to balance resource demand as it has
complete knowledge of all user's requirements and can flexibly change the amount of resources allocated to each
task. This can prevent bottlenecks through proper balances of resources allowing more tasks to run simultaneously
and can also lower the price due to there being a lower overall demand on resources.

Recently, cloud computing~\citep{cloud_cite} has become a popular solution for remotely running data-intensive
applications. But for some problem domains, it is not possible to use large cloud providers, for example running highly
delay-sensitive tasks or where connectivity to the cloud is intermittent.
Mobile edge computing~\citep{mobile_edge_survey} has emerged as a complementary paradigm to allow for small
data-centers, close to users, to execute tasks. These data centers are known as edge clouds.

Disaster response, smart cities and Internet-of-things (IoT) are popular technologies that utilise mobile edge
computing due to the use of ability to process small programs locally with low latency. For smart cities, this
allows for the possibly of smart intersections with the use of road-side sensors or smart traffic lights based
on cameras to minimise the waiting times~\citep{smart_cities_traffic_lights}. Or for the police to analysis
CCTV footage to spot suspicious behaviour or to track people between cameras~\citep{Sreenu2019}. In the case
of disaster response, maps can be produced using autonomous vehicles sensors to be used in the search for potential
victims and support responders~\citep{smart_disaster_management}.

To compute these task, several types of resources are required included communication bandwidth, computational power
and data storage resources~\citep{vaji_infocom}. Tasks will have a deadline such that the program must be completed
before this point and a private value. This value is depend on the program itself and its value to the owner,
.e.g analysis air pollution is less important than preventing traffic jams at rush hour or tracking a criminal on the
run. This project is interestedin allocated task to servers to maximise the social welfare (sum of all allocated
task values) over time. But due to users being self-interested, they may behave strategically~\citep{Bi2019} or prefer
to not reveal their value publicity~\citep{Pai2013}.

The shortcoming of existing work for resource allocation in edge cloud computing~\citep{vaji_infocom, Bi2019}
has the assumption that tasks have fixed resource requirements. However, flexibility is possible in practise
with how resources are allocaed to each task. For example, the allocated bandwidth for loading the program is
proportional to the time taken to load the program. This is true of also the computational requirements and
for sending results back to the user. This project investigates flexible allocation of resource and pricing
mechanisms when task arrive over time and have private values.