\chapter{Project problem}\label{ch:project-problem}
Cloud computing is a rapidly growing service with competition from Google, Amazon, Microsoft and more that aim to
allow users to run computer programs that are too large, difficult or time consuming to be run on a standard computer.
These services allow users to request a fixed amount of resources to run the program, e.g.\ cpu cores, RAM, hard drive
space, bandwidth, etc. However, as there are a limited amount of these resources available, bottlenecks can occur
when numerous users all require large amounts of resources. In turn, limiting the number of tasks
\footnote{Tasks, Programs and Jobs will be used interchangeable to refer to the same idea} that can be run in parallel.

This program is particular relevant in edge cloud computing as servers are small, commonly home desktops or servers,
with extremely limited resources in comparison to large data center where cloud computing is normally run. This means
that demand on resources is much greater making the process of allocating those resources and to which tasks is most
resource efficient an important research area.

This project considers the case where users state their total resource requirements for a program, instead of the
standard procedure that users request a fixed amount of resources for their program. This change now allows the cloud
provider the ability to balance resource demand as it had completed knowledge of user's requirements. With this
knowledge, a task's allocated resources can be dynamically changed as new tasks arrives or finish. Using this
alternative resource allocation procedure, bottlenecks can be prevent through proper balancing of resources that in
turn will allow more tasks to run simultaneously and can also lower the price to run a task.

While edge cloud computing is a developing paradigm~\citep{mobile_edge_survey} comparing to cloud computing, that has
become much more mainstream, edge cloud computing is believed to have a wide range of application where using standard cloud computing would be
impractical. This could be due to tasks being highly delay-sensitive, internet connectivity is intermittent or
operational security is high. \\
Currently: disaster response, smart cities and internet-of-things (IoT) are popular technologies that all utilise
cloud computing due to its ability to process programs locally with low latency. For smart cities, this
allows for the possibly of smart intersections with the use of road-side sensors or smart traffic lights based
on cameras to minimise the waiting times~\citep{smart_cities_traffic_lights}. Or for the police to analysis
CCTV footage to spot suspicious behaviour or to track people between cameras~\citep{Sreenu2019}. In the case
of disaster response, maps can be produced using autonomous vehicles sensors to be used in the search for potential
victims and support responders~\citep{smart_disaster_management}.

%% Todo not a fan currently of how this is written
To compute these task, several types of resources are required included communication bandwidth, computational power
and data storage resources~\citep{vaji_infocom}. Tasks also contain a deadline such that the program must be completed
before this point. Task will have a private monetary value that depends on the program itself and its value to the owner,
.e.g analysis air pollution is less important than preventing traffic jams at rush hour or tracking a criminal on the
run. This project is interested in allocated task to servers to maximise the social welfare (sum of all allocated
task values) over time. But due to users being self-interested, they may behave strategically~\citep{Bi2019} or prefer
to not reveal their value publicity~\citep{Pai2013}.

The shortcoming of existing work for resource allocation in edge cloud computing~\citep{vaji_infocom, Bi2019}
has the assumption that tasks have fixed resource requirements. However, flexibility is possible in practise
with how resources are allocated to each task. For example, the allocated bandwidth for loading the program is
proportional to the time taken to load the program. This is true of also the computational requirements and
for sending results back to the user. This project investigates flexible allocation of resource and pricing
mechanisms when task arrive over time and have private values.
